\documentclass[trans,compress,xcolor=table]{beamer}
\usepackage[utf8]{inputenc}
\usepackage{listings}
\usepackage{multicol}

\mode<article>
{
  \usepackage{fullpage}
  \usepackage{pgf}
  \usepackage{hyperref}
}

\mode<presentation>
{
  \usetheme{Antibes}
  \usecolortheme[rgb={0.9,0.4,0.1}]{structure}

  \setbeamercovered{transparent}
}


\title{Sofware Development with Scripting Languages:\\Browser Side Scripting}
\author{Onur Tolga Şehitoğlu}
\institute{Computer Engineering,METU}
\subject{Browser Side Scripting}
\date{2 January 2015}

\lstdefinelanguage{javascript}{
	morekeywords={
		break,const,continue,delete,do,while,export,for,in,function,
		if,else,import,in,instanceOf,label,let,new,return,switch,this,
		throw,try,catch,typeof,var,void,with,yield,class,extends
	},
	sensitive=false,
	morecomment=[l]{//},
	morecomment=[s]{/*}{*/},
	morestring=[b]",
	morestring=[d]'
}
		      

\begin{document}
\lstset{language=javascript,tabsize=4,
        basicstyle=\scriptsize\ttfamily,
        keywordstyle=\color{blue!50!black}\bfseries,
        identifierstyle=\color{blue!60!green},
        stringstyle=\color{red!70!green},
	commentstyle=\color{blue!30!white}\itshape,
        showstringspaces=true, escapeinside=`',columns=flexible, keepspaces=true}

\setbeamercolor{pexample}{bg=orange!5!white,fg=black}%


 \frame{\maketitle}
\begin{frame}
\begin{multicols}{2}
\tableofcontents
\end{multicols}
\end{frame}

\section{Javascript Syntax}
\begin{frame}
\frametitle{Javascript Syntax}
\begin{itemize}
\item Standard name is ECMAscript
\item C alike
\item Semicolon \lstinline!;! is optional for multiple lines
\item Interpreter is embedded on browser
\item Code can be embedded in \structure{HTML} with \lstinline!<script>!
	tag.
\item Browsers provide a \structure{Javascript} console
\end{itemize}
\end{frame}


\begin{frame}[fragile]
\begin{lstlisting}
<head>
<!-- include a script file from URL -->
<script src="/libs/jquery/1.7.1/jquery.min.js">
</script>
</head>
<body>
<script>
var msg = 'hello'; var target=' world'
alert(msg + target) {
}
</script>
\end{lstlisting}
\end{frame}

\section{Values and Types}
\begin{frame}[fragile]
\frametitle{Values and Types}
\begin{itemize}
\item Very minimal set of types: \lstinline!string, number, object!
\item Literals \lstinline!'Hello',"world", 3.454, 7, {name:'value',id:5}!
\item Type conversion is done as long as possible\\
\lstinline!a= "1"+4 ; a++ ; // a is 15!
\item Objects are created on the fly, in a way similar 
	to \structure{Python} 
	dictionaries. Object method can access instance with \lstinline!this!.
\begin{lstlisting}
var counter = { val: 0 , 
                get: function () {return this.val},
                incr: function() { this.val++}}

counter.get()
counter.incr()
\end{lstlisting}
\item No type enforcement on objects. 
\end{itemize}
\end{frame}

\section{Objects and Arrays}
\begin{frame}
\frametitle{Objects and Arrays}
\begin{itemize}
\item Object members \lstinline!student = \{ name: 'ali' \}! can be accessed in two ways:
\begin{itemize}
	\item \lstinline!student.name!
	\item \lstinline[]!student['name']!
\end{itemize}
\item \lstinline!Array! is a builtin object type that implements
	integer indexed elements.\\
	\lstinline[]!a = [7,18,23] ; a[0] = a[1] + a[2]!
\item Some array methods work in place, modifies array:\\
	\lstinline!a.sort() ; a.reverse()! \\
	\lstinline!a.push('hello') ; a.pop() ; a.unshift('hello') ; a.shift()!\\
\item Some returns new values:\\
	\lstinline!a.indexOf(3) ; a.join(":") ; a.slice(2,4); a.length!  \\
	\lstinline!a.filter(boolfunc) ; a.map(func); a.concat([3,4,5]) ; !\\
\end{itemize}
\end{frame}

\begin{frame}[fragile]
\frametitle{Manipulating Objects}
\begin{itemize}
\item Arbitrary members/methods can be added by assignments:\\
	\lstinline!student = {} ; student.name='Ali'!\\
	\lstinline!student['no'] = 444 // { name: 'Ali', no: 444 }!
\item Members can be deleted:\\
	\lstinline!delete student.name // { no: 444} !
\item Array elements can be deleted using 
	\lstinline!splice(start,num)!:\\
	\lstinline!a.splice(5,2)  // delete a[5], a[6]!
\item Assignment is by reference. Copy can be done member by member.
\item Testing for a member:\\
	\lstinline!if ( student.hasOwnProperty('name'))! or simply:\\
	\lstinline!if (student.name)!
\end{itemize}
\end{frame}

\section{Operators}
\begin{frame}
\frametitle{Operators}
\begin{itemize}
\item
Very similar to C. Most operators including  increments, arithmetic assignments and conditional expression
are works similar to C.
\item `\lstinline!+!' is concatenation unless both operands are numbers
\item \lstinline!/! is real division. \lstinline!Math.floor/celing/round()!
\item Numbers and strings are also objects. They have some
	conversion operators
\item \lstinline!Math! object contains more arithmetic functions.
\item \lstinline!/regexp/! creates a regular expression object.\\
	\lstinline!/^[0-9]+$/.test(mystring)! returns \lstinline!true!
	or \lstinline!false!, match result of regular expression 

\end{itemize}

\end{frame}


\section{Conditionals and Loops}
\begin{frame}[fragile]
\frametitle{Conditionals and Loops}
Similar to C:
\begin{itemize}
\item \lstinline!if (cond) else if (cond2) else (cond3);!
\item \lstinline!switch (val) { case num1:...; break; case "str1":...;...}!
\item \lstinline!while (cond) { ... }!
\item \lstinline!do { ... } while (cond)!
\item \lstinline!for (i=0; i < n; i++) { ... }!
\item Definite iteration on objects (\lstinline!a=[1,4,6,10]; b={x:3,y:4,z:5}!)\\
\lstinline!for (i in a) { ...  a[i] ...} //i=0,1,2,3!\\
\lstinline!for (i in b) { ...  a[i] ...} //i='x','y','z'!\\
\end{itemize}
\end{frame}


\section{Exceptions}
\begin{frame}[fragile]
\frametitle{Exceptions}
\begin{beamercolorbox}{pexample}
\begin{lstlisting}
try {
    ...
} catch (excp) {
    ...
} finally {
    ... // optional, executed always
}
\end{lstlisting}
\end{beamercolorbox}
\begin{itemize}
\item \lstinline!throw! statement can be used to raise an exception
\item If not handled exceptions stop current script execution
\end{itemize}
\end{frame}

\section{Defining Functions}
\begin{frame}[fragile]
\frametitle{Defining Functions}
\begin{beamercolorbox}{pexample}
\begin{lstlisting}
function f(param1, param2, param3)  {
    var x = ...

    if (param3 == undefined) param3 = 'default value'
    ...
   return val
}
// lambda like definition possible
a = [1,2,3,4]
b = a.map(function (x) { return x*x }   // b= [1,4,9,16]
// following two are equivalent:
function add(x,y) { return x+y } 
var add = function (x,y) { return x+y }
\end{lstlisting}
\end{beamercolorbox}
\end{frame}


\section{Variable Scope and Lifetime}
\begin{frame}[fragile]
\frametitle{Variable Scope and Lifetime}
\begin{itemize}
\item Variables are global by default, regardless of their
	position of initialization 
\item \lstinline!var! is used to create a variable in
	local scope.\\ \lstinline!var x; var y = 5! .\\
	\lstinline!for (var i = 0; i < n ; i++) ...! \\
\item Using always \lstinline!var! is a good practice. Otherwise
	using global scope causes confusions, especially in
	recursion.
\item Variables used by an object can be hidden using scope:
\begin{lstlisting}
function counter() {
    var val = 0; // hidden inside the function
    return { incr: function () { val++},
             get: function () { return val} }
}
a=counter()
a.incr() ;     console.log(a.get())
\end{lstlisting}
\end{itemize}
\end{frame}

\begin{frame}[fragile]
\begin{itemize}
\item Similar \structure{closures} can be defined.
\begin{lstlisting}
function multiplyby(x) {
    return function (y) {
        return x*y;
    }
}
twice = multiplyby(2);
twice(4); // will return 8
function sortby(field) {  
        function compare(a,b) { 
                return (a[field] < b[field])? -1 : 1
        }  
        return function (arr) { 
                arr.sort(compare)
        }
}
rlist = [{name:'z',no:43},{name:'a',no:31},{name:'c',no:11}]
namesort = sortby('name')  // a new function sorts by 'name'
nosort = sortby('no')	   // a new function sorts by 'no'
namesort(rlist)
nosort(rlist)
\end{lstlisting}
\end{itemize}
\end{frame}

\section{Assignment Semantics}
\begin{frame}
\frametitle{Assignment Semantics}
\begin{itemize}
\item Share semantics
\item Assignment copies reference, not data
\item Object assignment creates two variables denoting
	same object
\item Primitive values copied, objects shared (like Java)
\item Parameters pass by value for primitives, reference for
	objects
\item Copying requires special copy functions
\end{itemize}
\end{frame}

\section{Objects and Classes}
\begin{frame}[fragile]
\begin{itemize}
\item Objects can be created on the fly by \lstinline!{...}!
\item \lstinline!this! denotes the current object inside
	the function members.\\
	\lstinline!{c:0, incr: function() { this.c++}}!
\item Classes can be created by class functions updating members
	of \lstinline!this! and returning \lstinline!this!.
\item Updates on \lstinline!this! updates a function prototype
	and when \lstinline!return this! is executed this
	prototype instance is returned.
\item New instances can be created as \lstinline!new classfunction()!.
\item Define a binary search tree class \lstinline!BSTree! such that:
\begin{lstlisting}
bst = new BSTree()
f=[["orange","orange"],["apple","red"],["banana","yellow"],
   ["watermelon","green"],["strawberry","pink"]])
for (i in f) 	bst.insert(f[i][0], f[i][1])
color = bst.get("apple")
bst.forEach(function (n) { console.log(n.key+":"+n.value) })
\end{lstlisting}
\end{itemize}
\end{frame}
\begin{frame}[fragile]
\begin{lstlisting}
function BSTree() {
    var root = undefined;	// local scope, private 
    this.insert = function (key, value) {
        if (root == undefined) {
            root = { node: {key: key , value: value},
                     left: new BSTree(), right: new BSTree() }
        } else if ( key < root.node.key ) 
            root.left.insert(key, value)
        else if ( key > root.node.key )
            root.right.insert(key, value)
        else root.node.value = value;                     } 
    this.get = function (key) {
        if (root == undefined) return undefined
        else if (key < root.node.key) return root.left.get(key)
        else if (key > root.node.key) return root.right.get(key)
        else return root.node.value                      }
    this.forEach = function (func) {
        if (root == undefined)   return
        root.left.forEach(func)
        func(root.node)      // apply function to node content
        root.right.forEach(func)                        }
}
\end{lstlisting}
\end{frame}

\subsection{Function Prototypes and Inheritance}
\begin{frame}[fragile]
\frametitle{Function Prototype}
\begin{itemize}
\item Repeating function definitions in each instance is inefficient.
\item Javascript binding searches \lstinline!functionname.prototype! object
for missing members for the object. So that a single copy per function/class maintained
\begin{lstlisting}
function Counter() { this.value = 0 }

Counter.prototype.get = function () { return this.value }
Counter.prototype.incr = function () { this.value++ }

c = new Counter() ; d = new Counter() ; e = new Counter()
c.incr()
console.log(c.get())
\end{lstlisting}
\item \lstinline!c,d,e!  shares same prototype object of \lstinline!Counter!.
\item \alert{prototype modifications after construction of an object affects the object}
\end{itemize}
\end{frame}

\begin{frame}[fragile]
\lstinline!BSTree! class rewritten with prototype
\begin{lstlisting}
function BSTree() {
    this.node = undefined;	
}
BSTree.prototype.insert = function (key, value) {
    if (this.node == undefined) {
        this.node = {key: key , value: value}
        this.left = new BSTree(); this.right= new BSTree()
    } else if ( key < this.node.key ) 
        this.left.insert(key, value)
    else if ( key > this.node.key )
        this.right.insert(key, value)
    else this.node.value = value;                     } 
BSTree.prototype.get = function (key) {
    if (this.node == undefined) return undefined
    else if (key < this.node.key) return this.left.get(key)
    else if (key > this.node.key) return this.right.get(key)
    else return this.node.value                      }
BSTree.prototype.forEach = function (func) {
    if (this.node == undefined)   return
    this.left.forEach(func)
    func(this.node)      // apply function to node content
    this.right.forEach(func)                        }
\end{lstlisting}
\end{frame}

\begin{frame}[fragile]
\frametitle{Inheritance}
\begin{itemize}
\item Prototypes can be chained for inheritance. \lstinline!Object.create(protobject)!
Creates an object with prototype \lstinline!protobject!. If you set prototype of
\lstinline!B! as an object with same prototype with \lstinline!A!, bindings of \lstinline!B! will be chained to \lstinline!A!, results in inheritance.
\end{itemize}
\begin{lstlisting}
function Shape(x, y) { this.x = x ; this.y = y} 
Shape.prototype.move = function (x,y) { 
	this.draw('background'); this.x = x; this.y = y;
	this.draw('foreground') }

function Circle(x, y, r) { 
	Shape.apply(this,[x,y]); this.r = r   // Apply Shape constructor to this
}
Circle.prototype = Object.create(Shape.prototype)	// chain prototypes
Circle.prototype.constructor = Circle
Circle.prototype.draw = function (color) {
  console.log('drawing circle at', this.x, this.y, this.r, 'in', color)}

c = new Circle(10,10,5); c.move(30,20)
\end{lstlisting}
\end{frame}

\subsection{EcmaScript6 class}
\begin{frame}[fragile]
\begin{itemize}
\item Since EcmaScript 6 (2016) \lstinline!class! definitions are implemented as
syntactic sugar (maps code to old style prototype definitions)
\end{itemize}
\begin{lstlisting}
class Shape {
	constructor(x,y) { this.x = x; this.y = y}
    move(x,y) {  this.draw('background'); this.x = x; this.y =y ; 
				 this.draw('foreground')
	}
}
class Circle extends Shape {
	constructor(x,y,r) { super(x,y); this.r = r}
	draw(color) { 
	  console.log('drawing circle at', this.x, this.y, this.r, 'in', color)
	}
}
\end{lstlisting}
\end{frame}

\begin{frame}[fragile]
\lstinline!BSTree! class with new syntax
\begin{lstlisting}
class BSTree {
    constructor() { this.node = undefined}
	insert(key, value) {
    	if (this.node == undefined) {
        	this.node = {key: key , value: value}
        	this.left = new BSTree(); this.right= new BSTree()
    	} else if ( key < this.node.key ) this.left.insert(key, value)
    	else if ( key > this.node.key ) this.right.insert(key, value)
    	else this.node.value = value;                     
	}
	get (key) {
    	if (this.node == undefined) return undefined
    	else if (key < this.node.key) return this.left.get(key)
    	else if (key > this.node.key) return this.right.get(key)
    	else return this.node.value                     
	}
	forEach (func) {
    	if (this.node == undefined)   return
    	this.left.forEach(func)
    	func(this.node)      // apply function to node content
    	this.right.forEach(func)
	}
}
\end{lstlisting}
\end{frame}

\section{Browser Objects}
\begin{frame}[fragile]
\frametitle{Browser Objects}
\begin{itemize}
\item \lstinline!window!: an interface to browser. Loading a new page, etc.
\item \lstinline!document!: an interface to current document. Document Object Model members.
\item \lstinline!console!: Browser's Javascript console. 
\lstinline!console.log(message)! can be used to write debugging 
outputs on console.
\item \lstinline!document! can be used to access and update current
web page content dynamically.
\end{itemize}
\end{frame}

\subsection*{document Object}
\begin{frame}[fragile]
Most important members of \lstinline!document! are:
\begin{itemize}
\item \lstinline!document.head! : Header, \lstinline!<head>! element
	of current page.
\item \lstinline!document.body! : Body, \lstinline!<body>! element
\item \lstinline!document.forms! : list of \lstinline!<form>! elements in this document
\item \lstinline!document.title! : Page title.
\item \lstinline!document.cookie!: Cookies of current page.
\item \lstinline!document.links! : List of links of the document (as \lstinline!<a>! elements).
\item \lstinline!document.getElementById!: searches and returns 
the element with given \lstinline!id=...! attribute in the document.
\item \lstinline!document.getElementsByClassName! 
\lstinline!.getElementsByTagName!: searches and returns all elements with given \lstinline!class=...! attribute and given HTML tag respectively.
\item
\lstinline!document.children[0]!: Top element, \lstinline!<html>! of the document.
\end{itemize}
\end{frame}

\subsection*{DOM Elements}
\begin{frame}[fragile]
An HTML document contains nested elements. Top element is
\lstinline!<html>!. DOM element is a tree like representation
of document structure.
A DOM node named \lstinline!el! has the following members:
\begin{itemize}
\item \lstinline!el.children! : List of child HTML elements
	nested in this element. All \lstinline!<...></...>! elements
	that are direct child of \lstinline!el!.
\item \lstinline!el.childNodes!: List of child nodes including
	HTML elements and text.
\item \lstinline!el.nodeType!: Type of the node, 1 for element, 2 for attribute, 3 for text and 8 for comment node.
\item \lstinline!el.getAttribute(attr)!: returns attribute value for a
	given attribute. For example if an image element, \lstinline!el.getAttribute('src')! will give image URL.
\item \lstinline!el.id!: Element id by \lstinline!id="..."! attribute.
\item \lstinline!el.className!: element class by \lstinline!class="..."! attribute
\end{itemize}
\end{frame}

\begin{frame}[fragile]
\begin{itemize}
\item \lstinline!el.innerHTML!: HTML text content enclosed
	by element. Can be updated.
\item \lstinline!el.outerHTML!: HTML text content including
	the element. Can be updated. i.e. assigning it to \lstinline!''!
	deletes the element.
\item \lstinline!el.attributes!: Object to attributes. Attributes
	can be modified, inserted, removed.
\item \lstinline!el.setAttribute('border','1')!:
	sets an attribute of the element \\
\item \lstinline!el.removeAttribute('border')!:
	remove an attribute of the element \\
\item \lstinline!el.insertBefore(newel, child)!: 
	inserts an element as a child, before the child element.\\
	\lstinline!el.inserBefore(newel, el.children[0])! inserts
	it as the first child.
\item \lstinline!el.appendChild(newel)!:
	add element as the last child of the element.
\end{itemize}
\end{frame}

\begin{frame}[fragile]
\structure{Example:}
Add a \lstinline!<ul>! element to document at the end of
the document body:
\begin{lstlisting}
var li1 = document.createElement("LI")
var li2 = document.createElement("LI")
var li3 = document.createElement("LI")
li1.appendChild(document.createTextNode("apple"))
li2.appendChild(document.createTextNode("banana"))
li3.appendChild(document.createTextNode("orange"))
var ul = document.createElement("UL")
ul.appendChild(li1)
ul.appendChild(li2)
ul.appendChild(li2)
document.body.appendChild(ul)
\end{lstlisting}
\end{frame}

\begin{frame}[fragile]
\structure{Example:}
Recursively traverse all elements and log the tag name and
attributes:
\begin{lstlisting}
function traverse(el) {
    var str = el.tagName
    var attr = []
    if (el.attributes)
        for (var i=0 ; i < el.attributes.length; i++)
            attr.push(el.attributes[i].name + ":" + 
                    el.attributes[i].value)
    console.log(str + "[" + attr.join(",") + "]")
    // recurse to children
    if (el.children)
        for (var i=0; i < el.children.length ; i++)
            traverse(el.children[i])
}
traverse(document.body)
\end{lstlisting}
\end{frame}

\subsection*{DOM Actions}
\begin{frame}[fragile]
\frametitle{DOM Actions}
\begin{itemize}
\item HTML elements can have mouse, keyboard or other
browsers events handler functions.
\item Either defined as HTML attribute or set later on DOM:\\
	\lstinline!<img .. onclick="JS code here">!\\
	\lstinline!el.addEventListener("onclick" , function () { ... })!
\item Multiple event listeners can be added and remove on DOM:\\
	\lstinline!el.removeEventListener("onclick" , function () { ... })!
\item
\begin{itemize}
\item \lstinline!onclick!: mouse click
\item \lstinline!ondblclick!: mouse double click
\item \lstinline!onmouseover!: mouse pointer is over
\item \lstinline!onkeypress!: keyboard key pressed (event object paramater)
\item \lstinline!onbeforeload!: when object starts to be loaded
\item \lstinline!onload!: when object is loaded
\end{itemize}
\end{itemize}
\end{frame}

\begin{frame}[fragile]
\begin{itemize}
\item Events for form elements:
\begin{itemize}
\item \lstinline!onchange!: when form element content changed
\item \lstinline!onselect!: when form element is selected
\item \lstinline!onfocus!: when form element gets focus
\item \lstinline!onblur!: when form element loses focus
\item \lstinline!onsubmit!: when form is submitted. Function
	should return \lstinline!true! if default action (submit)
	should follow. Calling \lstinline!el.submit()! will
	explicitly submit the form data.
\end{itemize}
\item Form data can be validated through these events. Form elements
	give access to input elements with their names.\\
	\lstinline!document.forms[0].surname! gives element:\\
	\lstinline!<input ... name="surname"...>! of the first form.
\item Keyboard events get an event object containing \lstinline!keyCode, ctrlKey, shiftKey, AltKey! fields.
\item Mouse events get an event object containing \lstinline!button, clientX, clientY! fields. 
\end{itemize}
\end{frame}

\begin{frame}[fragile]
Form data validation example:
\begin{lstlisting}
<script>
function checkname(fname) {
        var n = document.getElementById('myform')
        var m = document.getElementById(fname + 'msg')
        if ( /^[a-z A-Z]+$/.test(n[fname].value)) {   
                m.textContent = "[OK]"; return true
        } else {
                m.textContent = "[INVALID]";  return  false
        } }
function checkage(fname) {
        var n = document.getElementById('myform')
        var m = document.getElementById(fname + 'msg')
        if ( n[fname].value > 0 && n[fname].value < 200) {      
                m.textContent = "[OK]"; return true
        } else {
                m.textContent = "[INVALID, 1-200]"; return  false
        } }
function checkform() {
        return (checkname('Name') && checkname('Surname') && 
                checkAge('Age')) }
</script>
\end{lstlisting}
\end{frame}

\begin{frame}[fragile]
Continued:
\begin{lstlisting}
...
<form id="myform" action="..." method="post">
Name: <input type="text" name="Name" 
             onblur="checkname('Name')"/>
      <span id="Namemsg"></span>
<br/>
Surname: <input type="text" name="Surname" 
                onblur="checkname('Surname')"/>
         <span id="Surnamemsg"></span>
<br/>
Age: <input type="text" name="Age" 
            onblur="checkage('Age')"/>
     <span id="Agemsg"></span>
<br/>
<input type="submit" name="submit" value="Send" 
       onclick="return checkform()">
</form>
...
\end{lstlisting}
\end{frame}

\section{Execution Model}
\begin{frame}
\frametitle{Execution Model}
\begin{itemize}
\item Javascript executes in a single thread/process (thread per browser tab/thread per nodejs connection)
\item Main interpreter code is an \structure{Event Loop} where a queue of messages are processed one message at a time. 
\item Each message in the queue is a Javascript function pointer. A new stack frame is created and function is called.
\item Browser pushes scripts in HTML code into the message queue on page load.
\item Each message is \structure{run to the completion}. Then next message is processed.
\item Browser events, timed events and callbacks push new messages.
\end{itemize}
\end{frame}

\begin{frame}[fragile]
\begin{itemize}
\item Execution is never preempted. Infinite loops and long running functions should be avoided.
\item Most I/O tasks works with call backs
\item Callback hell. Example: chain 3 functions with callbacks and possible error values:
\begin{lstlisting}
	getDocument(docA, vA => {
		if (vA) {
			getDocument(docB, vB => {
				if (vB) {
					getDocument(docC, vC => {
						if (vC) {
							// do something usefull with  vA + vB + vC
						} else { // handle error 
						}
					})
				} else { // handle error 
				}})
		} else { // handle error
		}})
\end{lstlisting}
\end{itemize}
\end{frame}

\begin{frame}[fragile]
\begin{itemize}
\item Run to completion is tricky:
\begin{lstlisting}
function getsum(userinput) {
	var sum = 0
	for (c of userinput) { // userinput is a list of items user selected
		makedbquery(c, v => {
			sum += v
		})} 
	return sum
}
console.log(getsum(['a','b','c']); // WON'T WORK!!!
\end{lstlisting}
\item  Solution: convert to callback style with recursion
\begin{lstlisting}
function getsum2(userinput, cb) {
	if (userinput.length == 0) { cb(0); }
	else {
		makedbquery(userinput[0], v => {
			getsum2(userinput.slice(1), sum => {
				cb(v + sum)
			})		
		})
	}
}
\end{lstlisting}
\end{itemize}
\end{frame}

\begin{frame}[fragile]
\frametitle{Promises}
\begin{itemize}
\item \structure{Promise} is a class defining an asynchronous event returning a value in future. Initially in pending state. When value is available, calls the  resolving action.
\begin{lstlisting}
function findPerson(name) {
	return new Promise(resolve => {
		getidfromname(name, v => resolve(v))
	 })
}
function getList(id) {
	return new Promise(resolve => {
		getclistfromid(id, v => resolve(v))
	})
}
findPerson('onursehitoglu')
	.then(getList)
	.then(lst => { console.log(lst) })
	.catch(err => { console.log(err)})
\end{lstlisting}
\item Promises give clear output
\end{itemize}
\end{frame}

\begin{frame}[fragile]
\begin{itemize}
\item Independent promises
\begin{lstlisting}
function getsum(userinput) {
	return new Promise(resolve => {
		// assume following give a list of promises
		var plist = userinput.map(makedbquery)
		Promise.all(plist).then( vlist => {
			// all promises resolved and gave a value list
			// resolve as the sum
			resolve(vlist.reduce( (v,r) => { return v+r}))
		})
	})
}
\end{lstlisting}
\end{itemize}
\end{frame}

\begin{frame}[fragile]
\frametitle{async/await}
\begin{itemize}
\item \structure{async}/\structure{await} keywords make
asynchronous programming much more easier.
\item define all asynchronous functions with \structure{async} prefix
\item call all asynchrouse functions with \structure{await} prefix
\item rest of the function is put into a message (callback) and inserted in queue when asynchronous function completes.
\item Javascript make all necessary conversions.
\begin{lstlisting}
async function wait(t) {
	return new Promise(resolve => {
		setTimeout(resolve, t) })
}
main = async () => {
	await wait(1000);
	console.log('hello')
	await wait(1000);
	console.log('world')
}
main()
\end{lstlisting}
\end{itemize}
\end{frame}

\section{AJAX}
\begin{frame}
\frametitle{AJAX}
\begin{itemize}
\item \structure{A}synchronous \structure{J}avascript \structure{A}nd \structure{X}ML 
\item Browser script can make HTTP requests to web server to 
load/update data dynamically.
\item \lstinline!XMLHttpRequest! object does the job.
\item Synchronous operation: block until request completed and result is taken.
	\textbf{blocks interpreter!}. 
\item Asynchronous operation: gets a callback function and function is called
	by browser when request status changes.
\item \lstinline!open(method,path, async)! method sets up the connection
\item \lstinline!send()! will initiate the request
\item On success \lstinline!response! contains the server response.
\item If server sends XML document \lstinline!responseXML! contains DOM of it.
\end{itemize}
\end{frame}

\begin{frame}[fragile]
\begin{lstlisting}
function getFlights(depcity) {
    var req = new XMLHttpRequest()
    // assume getflights/city gives direct flights
    req.open('POST','/getflights/'+depcity, false) // syncronous
    req.send('id=all&nres=100')	  // blocks until response ready
    if (req.status == 200) { // success
        return req.responseXML    // XML DOM of result
    } else 
        return undefined
}
\end{lstlisting}
Server can return an XML DOM value of direct flights from city:
\begin{lstlisting}[language=XML]
<xml>
<flights><flight no="LH1231" dep="ESB" dest="MUN">Munih</flight>
         <flight no="TK5434" dep="ESB" dest="ERC">Lefkosa</flight>
         <flight no="TK22" dep="ESB" dest="BRL">Berlin</flight>
</flights>
</xml>
\end{lstlisting}
This DOM can be processed similar to HTML DOM.
\end{frame}

\begin{frame}[fragile]
Asynchronous operation handled by \lstinline!onreadystatechange! event,
called on different stages. \lstinline!readyState == 4! 
when connection is ended. 
\begin{lstlisting}
function getFlights(depcity, updatefunc) {
    var req = new XMLHttpRequest()
    req.open('POST','/getflights/'+depcity, true) // asyncronous
    req.onreadystatechange = function () {
		if (req.readyState != 4) return; // intermediate states
        if (req.status == 200) { // success
            updatefunc(req.responsXML)    
        } else alert('request failed:' + req.status)		 } 
    req.send('id=all&nres=10')   // if POST, send form data here
}// no value returned. function will be called when ready
function updflight(xmldom) {   // update <SELECT> with flights
    var flghts = xmldom.getElementsByTagName('flight')
    var selbox = document.getElementById('flightselect')
    selbox.innerHTML = ""
    for (var i in flghts) {
        var fno = flghts[i].getAttribute('no')
        var fdest = flghts[i].getAttribute('dest')
        var opt = document.createElement("OPTION")
        opt.setAttribute('fno', fno)
        opt.textContent = flghts[i].textContent
    } 			}
getFligths('ankara',updflight)
\end{lstlisting}
\end{frame}

\subsection*{JSON}
\begin{frame}[fragile]
\frametitle{JSON}
\begin{itemize}
\item \structure{J}ava\structure{S}cript \structure{O}bject \structure{N}otation
\item Shorter and easier then XML for exchanging data. Similar to JavaScript.
\begin{lstlisting}[showstringspaces=false]
user = '{"name":"Onur", 
         "courses": [{"code":5710498, "name":"ceng498"},
                     {"code":5710536, "name":"ceng536"}],
         "room":"B-209"}'           
// eval('userobj = ' + user)   works but insecure!!!! Avoid it....
userobj = JSON(user)		// Javascript object
userstring = JSON(userobj)	// back to first string
\end{lstlisting}
\item All languages have libraries for JSON generation and parsing.
\begin{lstlisting}[language=python,showstringspaces=false]
import json
user = '{"name":"Onur", 
         "courses": [{"code":5710498, "name":"ceng498"},
                     {"code":5710536, "name":"ceng536"}],
	 "room":"B-209"}'           
userobj = json.loads(user)	# python dictionary
userstring = json.dumps(userobj)  # back to string
\end{lstlisting}

\end{itemize}
\end{frame}

\begin{frame}[fragile]
\begin{itemize}
\item JSON is denoted by \lstinline!application/json! in 
	\lstinline!Content-type! header.
\begin{lstlisting}
Content-type: application/json

{"result":"success",
 "flights":[ 
    {"no":"LH1231","dep":"ESB","dest":"MUN","city":"Munih"},
    {"no":"TK5434","dep":"ESB","dest":"ERC","city":"Lefkosa"},
    {"no":"TK2222","dep":"ESB","dest":"BRL","city":"Berlin"}
  ]
}
\end{lstlisting}
\item Server side can easily convert their objects into JSON strings.
\item \lstinline!XMLHttpRequest! object \lstinline!response!
	member can be parsed directly into Javascript object.\\
 \lstinline!result = JSON.parse(x.response)!
\item Much more easier than traversing XML DOM.
\end{itemize}
\end{frame}

\section{JQuery}
\begin{frame}
\frametitle{JQuery}
\begin{itemize}
\item A Javascript library for easier Javascript development
\begin{itemize}
\item Shorter and easy to use element selectors \lstinline!$(".id")!
\item Browser version abstraction. Same code support multiple browsers
\item Animations with stylesheet
\item Easy document manipulation and AJAX
\item Ready to use interactive widgets through \structure{JQuery-UI} 
\item Many contributed applications and widgets sets.
\end{itemize}
\item You need to download a version from \url{http://jquery.com} and
	include in web page for example inside of \lstinline!<head></head>!:\\
	\lstinline!<script src="/jquery/jquery.min.js"></script>"!
\item or Google hosted one:\\
	\lstinline!<script src="//ajax.googleapis.com/ajax/libs/jquery/1.9.1/jquery.min.js">!
\end{itemize}
\end{frame}

\subsection*{Selectors}
\begin{frame}[fragile]
\frametitle{Selectors}
\begin{itemize}
\item \lstinline!$! is a shorthand for \lstinline!jQuery!.
\item Selector \lstinline!$()! is entry point of JQuery.
\item Selector select a set of DOM elements. Returns a result object.\\
	\lstinline!$("tagname"), $("#id"), $(".class"), $("parent > immchild")!\\
	\lstinline!$("parent descend") $("[attr]"), $("[attr=value]") !\\
	\lstinline{$("[attr!=notvalue]") $("[attr^=prefix]"), $("[attr$=suffix]") }\\
	\lstinline{$("p,span,div")} (or conn.), 
\item There are more in selector syntax and selectors can be combined:\\
\lstinline!$("table.plist img[src $= jpg]")! selects all \lstinline!img! tags
	with \lstinline!src! attribute ends with \lstinline!jpg! and that
	are contained in tables with class \lstinline!plist!.
\end{itemize}
\end{frame}

\subsection*{Manipulation}
\begin{frame}[fragile]
\frametitle{Manipulation}
\begin{itemize}
\item Matched elements inner content can be changed. Operation is applied on
all elements:
\begin{itemize}
\item \lstinline!.html(content)! replaces inner HTML.
\item \lstinline!.text(content)! replaces inner text, HTML escaped.
\item \lstinline!.append(content)! add content as the last item inside
\item \lstinline!.before(content)! add content as the first item inside
\item \lstinline!.wrapInner(content)! enclose inner elements with content
\end{itemize}
\item Matched elements surroundings can be changed:
\begin{itemize}
\item \lstinline!.wrap(content)! elements enclosed in content
\item \lstinline!.after(content)! content inserted after elements
\item \lstinline!.before(content)! content inserted before elements
\end{itemize}
\item Elements can be removed or hidden
\begin{itemize}
\item \lstinline!.remove()! remove elements
\item \lstinline!.hide(), show(), toggle()! hide, show, toggle visibility
\end{itemize}
\item Attributes:
\begin{itemize}
\item \lstinline!.addClass(cls), removeClass(cls)! 
\item \lstinline!.attr(attr,val), .removeAttr(attr)! {\scriptsize set/remove elements attribute}
\end{itemize}
\end{itemize}
\end{frame}

\subsection*{Traversal}
\begin{frame}[fragile]
\frametitle{Traversal}
\begin{itemize}
\item \lstinline!.find(sel)! apply filter on currenet elements
\item \lstinline!.each(func)! apply function to all elements\\
	\lstinline!$("*").each(function (i,e){console.log(i+el.tagName}))! \\
	will print all element tags in document.
\item \lstinline!.next()! give sibling following each element
\item \lstinline!.prev()! give sibling preceding each element
\item \lstinline!.children()! give children of each element
\item \lstinline!.parent()! give parent of each element
\end{itemize}
\end{frame}


\subsection*{Events}
\begin{frame}[fragile]
\frametitle{Events}
\begin{itemize}
\item General form of setting an event handler:\\
	\lstinline!.on("event", data, func (ev) {...})!\\
	\lstinline!data! is passed as \lstinline!ev.data! to handler.\\
	\lstinline!.on("event", func (ev) {...})! is no data version\\
	\lstinline!event! can be \lstinline!"click", "blur", "focus","submit", "select",...!
\item All events are also provided as functions:\\
	\lstinline!.click(func), .blur(funct), .submit(funct),...!
\item Cancelling event handlers: \lstinline!.off()!, all handlers.\\
	 \lstinline!.off("click")!, all click handlers.\\
	\lstinline!.off("click",myfunc)! only \lstinline!myfunc! handler.
\item \lstinline!.one()! binds a one time only event handler.
\end{itemize}
\end{frame}


\subsection*{AJAX}
\begin{frame}[fragile]
\begin{itemize}
\item \lstinline!$.post()! and \lstinline!$.get()! can be used to
	make AJAX requests directly.
\item \lstinline!$.post(url, data, succfunc, datatype)! is the full usage:\\
\begin{itemize}
\item \lstinline!data! is either a query string or an object. If object
	\structure{jQuery} converts it into a query string using 
	\lstinline!$.param()!
\item \lstinline!succfunc! is a function to be called on success. It gets
	server response as parameter.
\item \lstinline!datatype! is the type of serer response, xml,json,...
\item \lstinline!$.post(...).fail(func(){..})! defines failure functions
\end{itemize}
\item Form data can be converted on POST query string:\\
	\lstinline!$('form').serialize()!
\item For detailed functionality, use \lstinline!$.ajax()!.
\end{itemize}
\end{frame}

\begin{frame}[fragile]
Flights example with \structure{jQuery}:
\begin{lstlisting}
function getFlights(depcity, updatefunc) {
    $.get('/getflights/'+depcity, updflight)
     .fail(function() { alert('error')})
    // assume JSON response, jQuery automatically parse
    // and send object to success function
}

function updflight(resp) {   // assume json respons
    $('#flightselect').html('')          // clear
    for (var i in resp.flights) {
        $('#flightselect').append('<option></option')
        // :last() select last child. methods can be cascaded
        $('#flightselect>:last()').attr('fno',resp.flights[i].fno)
                                  .text(resp.flights[i].city)
    }
}
getFligths('ankara',updflight)
\end{lstlisting}
\end{frame}

\section{Dynamic Web Applications}
\begin{frame}[fragile]
\begin{itemize}
\item Implement \structure{View} on browser side
\item \structure{Control} on browser side call server side 
	\structure{Control} via \structure{AJAX}
\item Server side model updates are loaded in view on browser side.
\item All server responses converted into XML or more practically JSON
\item Javascript get responses and update web page.
\item More dynamism achieved in HTML5 through \structure{web sockets}.
\end{itemize}
\end{frame}
\end{document}

